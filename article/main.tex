\documentclass[conference]{IEEEtran}
\usepackage{blindtext, graphicx}

\usepackage[slovak]{babel}
\usepackage[utf8]{inputenc}
\usepackage{cite}

\usepackage{url}
\usepackage{amsmath}

\usepackage{algorithm}
\usepackage{algpseudocode}

\algnewcommand\algorithmicforeach{\textbf{for each}}
\algdef{S}[FOR]{ForEach}[1]{\algorithmicforeach\ #1\ \algorithmicdo}

% correct bad hyphenation here
\hyphenation{op-tical net-works semi-conduc-tor}

\begin{document}
%
% paper title
% can use linebreaks \\ within to get better formatting as desired
\title{Porovnanie času konvergencie siete medzi softvérovo riadenými sieťami a sieťami podporujúcimi protokol OSPF}

% author names and affiliations
% use a multiple column layout for up to three different
% affiliations
\author{\IEEEauthorblockN{Peter Kaňuch}
\IEEEauthorblockA{Fakulta informatiky a informačných technológií\\
Slovenská Technická Univerzita\\
Slovenská republika, 841 04 Bratislava IV\\
Email: xkanuch@stuba.sk}\and
\IEEEauthorblockN{Nikolas Janec}
\IEEEauthorblockA{Fakulta informatiky a informačných technológií\\
Slovenská Technická Univerzita\\
Slovenská republika, 841 04 Bratislava IV\\
Email: xjanec@stuba.sk}}



% use for special paper notices
%\IEEEspecialpapernotice{(Invited Paper)}

% make the title area
\maketitle

\begin{abstract}
%\boldmath

\end{abstract}

\begin{IEEEkeywords}
SDN, Mininet, OpenFlow, HP Van, OSPF, Cisco iOS
\end{IEEEkeywords}

\IEEEpeerreviewmaketitle

\section{Úvod}

Najčastejšie používaným smerovacím protokolom vo veľkých korporátnych, či firemných sieťach je protokol OSPF (\textit{Open Shortest Path First}) z dôvodu, že tento protokol podporuje širokú škálu funkcionality pre siete poskytovateľov \cite{first}. V poslednom období sa začali dostávať do popredia SDN siete (\textit{Softvérovo riadené siete}), ktoré prinášajú určité výhody oproti klasickým sieťam.

\section{OSPF}

Protokol OSPF patrí do skupiny protokolov \textit{Link-state}, a preto každý smerovač zapojený do siete pozná všetky ostatné smerovače v sieti, vzájomné prepojenia medzi nimi, všetky koncové aj prepojovacie siete a ceny všetkých rozhraní.


\section{Softvérovo riadené siete}

\subsection{SDN}
Softvérovo riadené siete (SDN) majú centralizovanú architektúru \cite{second}, čo znamená, že zariadenia v sieti sú centralizovane riadené z jedného kontroléra. SDN architektúru tvoria nasledovné vrstvy:

\begin{itemize}
	\item{Aplikačná vrstva - ponúka služby pre virtualizáciu, smerovanie, firewall, ...}
	\item{Kontrolná vrstva - pozostáva z kontrolera, ktorý jednak komunikuje s aplikáciami cez rozhranie, ale aj priamo s fyzickou vrstvou (infraštruktúrou).}
	\item{Vrstva infraštruktúry - pozostáva z fyzických zariadení pripojených v sieti}
\end{itemize}

Základným prvkom v Softvérovo riadených sieťach je virtualizácia. Tá dovoľuje, aby softvér, ktorý riadi celú sieť bežal nezávisle od fyzickej vrstvy (zariadení/hardvéru) \cite{third}.

\begin{figure}[h!]
\centering
\includegraphics[width=3in]{../img/sdn-architecture}
\caption{SDN architektúra}
\end{figure}

\subsection{OpenFlow}

Komunikačný protokol pre SDN siete, ktorý umožňuje priamu interakciu kontrolera s fyzickými zariadeniami. Je štandard, ktorý dovoľuje vzdialene konfigurovať zariadenia od rôznych výrobcov \cite{fourth}. Taktiež umožňuje úpravu smerovacích tabuliek pomocou pridávania rôznych pravidiel.

\section{Testovanie}

\subsection{Testovacie prostredie}

\begin{itemize}
\item{VirtualBox v5.1.28}
\item{Ubuntu-14.04.5-server-amd64}
\item{Mininet simulátor -}
Mininet je nástroj určený pre simuláciu SDN sieti \cite{fifth}. Napodobňuje kompletnú sieť zariadení ako hosty, prepínače a jednotlivé prepojenia medzi nimi. Dokáže simulovať sieť pomocou virtualizácie založeniej na procesoch. Taktiež podporuje protokol OpenFlow.
\item{HP Van SDN kontrolér -} je softvér, ktorý poskytuje centrálny bod pre správu sieti podporujúce protokol OpenFlow \cite{sixth}. Taktiež poskytuje rozhranie pre vývoj Java aplikácií inými vývojarmi. Je navrhnutý najmä pre fungovanie v dátových centrách.

\begin{figure}[h!]
\centering
\includegraphics[width=3in]{../img/HPVAN}
\caption{HP Van SDN kontrolér}
\end{figure}


\item{Python}
\item{Cisco iOS 7200}
\item{GNS3 -} nástroj určený pre simuláciu sieti podporujúcich Cisco zariadenia.
\item{TCPping -} nástroj, ktorý dokáže simulovať HTTP komunikáciu.
\end{itemize}

\subsection{Topológia siete}

Pre testovacie účely sme vytvorili tri veľkostí topológie, a to topológia s desiatimi, dvadsiatimi a tridsiatimi zariadeniami. 

\begin{figure}[h!]
\centering
\includegraphics[width=3in]{../img/topology}
\caption{Príklad topológie s dvadsiatimi zariadeniami v nástroji GNS3}
\end{figure}

\begin{figure}[h!]
\centering
\includegraphics[width=3in]{../img/topologySDN}
\caption{Príklad topológie s desiatimi zariadeniami v Mininete}
\end{figure}

\subsection{Konvergencia siete}

\subsection{Doba odozvy}


\subsection{}



\section{}



\section{}







% Can use something like this to put references on a page
% by themselves when using endfloat and the captionsoff option.
\ifCLASSOPTIONcaptionsoff
  \newpage
\fi


\begin{thebibliography}{1}
\bibitem{first}KHAN, Asad Ali, et al. A Convergence Time Optimization Paradigm for OSPF based Networks Through SDN SPF Protocol Computer Communications and Networks (CCN)/Delay Tolerant Networks. In: Proceedings of the International Conference on Future Networks and Distributed Systems. ACM, 2017. p. 43.

\bibitem{second}Software-Defined Networking (SDN) Definition, \url{https://www.opennetworking.org/sdn-definition/}, Accessed: 19.11.2017

\bibitem{third}FOGARTY, Susan. 7 Essentials of Software-Defined Networking. Information Week Network Computing, 2013.

\bibitem{fourth}What is OpenFlow? Definition and How it Relates to SDN, \url{https://www.sdxcentral.com/sdn/definitions/what-is-openflow/},  Accessed: 19.11.2017

\bibitem{fifth}Mininet: Rapid Prototyping for Software Defined Networks, \url{https://github.com/mininet/mininet}

\bibitem{sixth}HP Virtual Application Networks (VAN) SDN Controller, \url{https://www.sdxcentral.com/products/hp-virtual-application-networks-van-sdn-controller/}

\end{thebibliography}






\end{document}


